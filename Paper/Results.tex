\section{Results}\label{sec:resultados}
Five pseudo chaotic maps were studied. For each one a floating point representation, a decimal numbers representation with $1\leq P \leq 27$ and a binary numbers representation with $1\leq B \leq 27$ are considered. For each representation $1000$ time series were generated using randomly chosen initial conditions within the interval $[0,1]$. 
The studied maps are tent (TENT), logistic (LOG) a sequential switching between TENT and LOG (SWITCH). Furthermore a skipping randomization procedure is applied to SWITCH \cite{DeMicco2008}, discarding the values in the odd positions (EVEN) or the values in the even positions (ODD) respectively. Let us detail our results for each of these maps.   

\subsection {Simple maps.}\label{subsec:SimpleMaps}
Here we report our results for both maps:

\input{LogMapResults}

\input{TentMapResults}

In summary, a comparison between LOG and TENT maps shows that only LOG  can be used because TENT is highly anomalous. 

\subsection{Sequential switching}\label{subsec:SecSwitch}

\input{SwitchMapResults}

\input{SkippingResults}

\input{PeriodResults}